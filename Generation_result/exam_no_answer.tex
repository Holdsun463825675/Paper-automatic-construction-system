\documentclass[12pt]{article}
\usepackage{ctex}
\usepackage{geometry}
\usepackage{enumitem}
\geometry{a4paper, margin=2.5cm}
\begin{document}
\begin{center}
\LARGE\textbf{编译原理试题}
\end{center}
\vspace{0.5cm}
这是编译原理试题。
\vspace{0.5cm}
\section*{一、选择题(共2题,每题10分,共20分)}
\noindent\textbf{1.} 编译的第一阶段是:\\[0.2em]
\begin{itemize}[label=~]
\item[\textbf{A.}] 词法分析
\item[\textbf{B.}] 语法分析
\item[\textbf{C.}] 生成目标代码
\item[\textbf{D.}] 代码优化
\end{itemize}
\vspace{0.5cm}
\noindent\textbf{2.} 以下哪种数据结构适合符号表?\\[0.2em]
\begin{itemize}[label=~]
\item[\textbf{A.}] 数组
\item[\textbf{B.}] 队列
\item[\textbf{C.}] 哈希表
\item[\textbf{D.}] 栈
\end{itemize}
\vspace{0.5cm}
\section*{二、填空题(共1题,每题10分,共10分)}
\noindent\textbf{3.} 编译器前端主要包括\_\_\_\_\_\_\_\_和\_\_\_\_\_\_\_\_。\\[0.2em]
\vspace{0.5cm}
\section*{三、判断题(共1题,每题10分,共10分)}
\noindent\textbf{4.} 编译器的后端生成目标代码。\\[0.2em]
\vspace{0.5cm}
\end{document}